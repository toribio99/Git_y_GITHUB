% !TEX program = lualatex
\documentclass[12pt,a4paper]{article}

% Preámbulo universal con detección de motor
\usepackage{iftex}

\ifluatex
  \usepackage{fontspec}
  \usepackage{polyglossia}
  \setdefaultlanguage[variant=mexican]{spanish}
\else
  \ifxetex
    \usepackage{fontspec}
    \usepackage{polyglossia}
    \setdefaultlanguage[variant=mexican]{spanish}
  \else
    \usepackage[T1]{fontenc}
    \usepackage[spanish,es-tabla]{babel}
  \fi
\fi

\usepackage{geometry}
\geometry{margin=2.5cm}
\usepackage{xcolor}
\usepackage{listings}
\usepackage{tcolorbox}
\tcbuselibrary{listings,skins,breakable}
\usepackage{hyperref}
\hypersetup{colorlinks=true,linkcolor=blue,urlcolor=blue}

% Configuración de listings para Bash
\lstdefinestyle{bashstyle}{
  language=bash,
  basicstyle=\ttfamily\small,
  keywordstyle=\color{blue}\bfseries,
  commentstyle=\color{gray}\itshape,
  stringstyle=\color{red},
  showstringspaces=false,
  breaklines=true,
  frame=single,
  backgroundcolor=\color{gray!10},
  numbers=none
}

% Estilos de cajas
\newtcolorbox{infobox}[1][]{
  colback=blue!5!white,
  colframe=blue!75!black,
  fonttitle=\bfseries,
  title=#1,
  breakable
}

\newtcolorbox{warningbox}[1][]{
  colback=orange!5!white,
  colframe=orange!75!black,
  fonttitle=\bfseries,
  title=#1,
  breakable
}

\newtcolorbox{successbox}[1][]{
  colback=green!5!white,
  colframe=green!75!black,
  fonttitle=\bfseries,
  title=#1,
  breakable
}

\title{\Huge\bfseries Guía Completa:\\Git y GitHub\\para Proyectos LaTeX}
\author{\Large Toribio de J. Arrieta F.}
\date{\today}

\begin{document}

\maketitle
\tableofcontents
\newpage

%━━━━━━━━━━━━━━━━━━━━━━━━━━━━━━━━━━━━━━━━━━━━━━━━━━━━━━━━━━━━━━━━━━━━━━━━━━━━━
\section{Introducción}
%━━━━━━━━━━━━━━━━━━━━━━━━━━━━━━━━━━━━━━━━━━━━━━━━━━━━━━━━━━━━━━━━━━━━━━━━━━━━━

\subsection{¿Qué es Git?}

\textbf{Git} es un sistema de control de versiones distribuido que permite:
\begin{itemize}
  \item Guardar el historial completo de cambios de tus archivos
  \item Volver a versiones anteriores cuando lo necesites
  \item Trabajar sin internet (todo se guarda localmente)
  \item Colaborar con otras personas en el mismo proyecto
\end{itemize}

\subsection{¿Qué es GitHub?}

\textbf{GitHub} es una plataforma en la nube que:
\begin{itemize}
  \item Aloja tus repositorios Git en internet
  \item Sirve como respaldo de tu trabajo
  \item Permite acceder a tus proyectos desde cualquier computadora
  \item Facilita la colaboración y el código abierto
\end{itemize}

\subsection{¿Por qué usar Git + GitHub para LaTeX?}

\begin{successbox}[Beneficios para proyectos LaTeX]
\begin{enumerate}
  \item \textbf{Nunca más perderás trabajo}: Cada versión queda guardada
  \item \textbf{Recupera versiones antiguas}: Si borras algo por error, puedes recuperarlo
  \item \textbf{Respaldo en la nube}: Si tu computadora se daña, tus archivos están seguros
  \item \textbf{Trabajo offline}: Git funciona sin internet, solo necesitas conexión para sincronizar
  \item \textbf{Historial completo}: Ves qué cambiaste, cuándo y por qué
\end{enumerate}
\end{successbox}

%━━━━━━━━━━━━━━━━━━━━━━━━━━━━━━━━━━━━━━━━━━━━━━━━━━━━━━━━━━━━━━━━━━━━━━━━━━━━━
\section{Configuración Inicial}
%━━━━━━━━━━━━━━━━━━━━━━━━━━━━━━━━━━━━━━━━━━━━━━━━━━━━━━━━━━━━━━━━━━━━━━━━━━━━━

\subsection{Instalación de GitHub CLI}

El GitHub CLI (\texttt{gh}) es la herramienta oficial para interactuar con GitHub desde la terminal.

\begin{lstlisting}[style=bashstyle]
# Instalar con Homebrew
brew install gh
\end{lstlisting}

\subsection{Autenticación con GitHub}

\begin{lstlisting}[style=bashstyle]
# Iniciar autenticación
gh auth login
\end{lstlisting}

\textbf{Pasos que verás:}
\begin{enumerate}
  \item Te dará un código de un solo uso (ej: \texttt{BC0B-57BD})
  \item Te pedirá abrir: \url{https://github.com/login/device}
  \item Pega el código en la página web
  \item Autoriza GitHub CLI
  \item ¡Listo! Ya estás conectado
\end{enumerate}

\subsection{Verificar autenticación}

\begin{lstlisting}[style=bashstyle]
# Ver estado de autenticación
gh auth status
\end{lstlisting}

Deberías ver algo como:
\begin{lstlisting}[style=bashstyle]
github.com
  ✓ Logged in to github.com account toribio99 (keyring)
  - Active account: true
\end{lstlisting}

%━━━━━━━━━━━━━━━━━━━━━━━━━━━━━━━━━━━━━━━━━━━━━━━━━━━━━━━━━━━━━━━━━━━━━━━━━━━━━
\section{Conceptos Fundamentales}
%━━━━━━━━━━━━━━━━━━━━━━━━━━━━━━━━━━━━━━━━━━━━━━━━━━━━━━━━━━━━━━━━━━━━━━━━━━━━━

\subsection{Vocabulario básico}

\begin{description}
  \item[Repositorio (repo)] Carpeta que contiene tu proyecto y todo su historial de cambios
  \item[Commit] ``Fotografía'' o punto de guardado de tu proyecto en un momento específico
  \item[Staging Area] Área de preparación donde pones archivos antes de hacer commit
  \item[Push] Subir tus commits a GitHub (requiere internet)
  \item[Pull] Descargar cambios de GitHub a tu computadora
  \item[Clone] Copiar un repositorio de GitHub a tu computadora
  \item[Remote] Conexión entre tu repositorio local y GitHub
\end{description}

\subsection{El flujo de trabajo básico}

\begin{infobox}[Ciclo típico de trabajo]
\begin{enumerate}
  \item \textbf{Editas} tus archivos .tex normalmente
  \item \textbf{git add} --- Agregas los cambios al área de preparación
  \item \textbf{git commit} --- Guardas los cambios con un mensaje descriptivo
  \item \textbf{git push} --- Subes a GitHub (cuando tengas internet)
\end{enumerate}
\end{infobox}

\subsection{Concepto crucial: Git NO sincroniza automáticamente}

\begin{warningbox}[¡MUY IMPORTANTE!]
\textbf{Pregunta frecuente:} ``Si edito un archivo en \texttt{LaTeX-GitHub/}, ¿se actualiza automáticamente en GitHub y en \texttt{03\_LaTex/}?''

\textbf{Respuesta:} \textbf{NO.} Nada es automático. Los tres lugares son COMPLETAMENTE INDEPENDIENTES.
\end{warningbox}

\subsubsection{Los tres lugares de almacenamiento}

Cuando trabajas con Git y GitHub, tus archivos pueden existir en hasta 3 lugares diferentes:

\begin{enumerate}
  \item \textbf{Carpetas originales} (ej: \texttt{03\_LaTex/})
  \begin{itemize}
    \item Tus archivos originales
    \item Pueden tener Git local
    \item NO están conectados a GitHub
    \item Los cambios aquí NO van a ningún lado automáticamente
  \end{itemize}

  \item \textbf{Carpetas clonadas} (ej: \texttt{LaTeX-GitHub/})
  \begin{itemize}
    \item Copias descargadas desde GitHub
    \item SÍ están conectadas a GitHub
    \item Los cambios aquí NO se suben automáticamente
    \item Debes hacer \texttt{git add}, \texttt{git commit}, \texttt{git push} manualmente
  \end{itemize}

  \item \textbf{GitHub} (la nube)
  \begin{itemize}
    \item Tus repositorios en internet
    \item Solo se actualiza cuando haces \texttt{git push}
    \item Requiere internet
  \end{itemize}
\end{enumerate}

\subsubsection{¿Qué pasa cuando editas un archivo?}

\begin{warningbox}[Escenario A: Editas en LaTeX-GitHub/]
\textbf{Si editas:} \texttt{LaTeX-GitHub/LaTeX-Varios/archivo.tex}

\textbf{Resultado:}
\begin{itemize}
  \item ❌ NO aparece automáticamente en \texttt{03\_LaTex/}
  \item ❌ NO sube automáticamente a GitHub
  \item ✅ SOLO cambia en \texttt{LaTeX-GitHub/}
\end{itemize}

\textbf{Para subir a GitHub:}
\begin{lstlisting}[style=bashstyle]
git add .
git commit -m "mensaje"
git push  # Solo aquí sube a GitHub
\end{lstlisting}
\end{warningbox}

\begin{warningbox}[Escenario B: Editas en 03\_LaTex/]
\textbf{Si editas:} \texttt{03\_LaTex/Practicas De Latex/archivo.tex}

\textbf{Resultado:}
\begin{itemize}
  \item ❌ NO aparece en \texttt{LaTeX-GitHub/}
  \item ❌ NO sube a GitHub (no está conectado)
  \item ✅ SOLO cambia en \texttt{03\_LaTex/}
\end{itemize}

\textbf{Esta carpeta NO está conectada a GitHub}, así que aunque hagas commit, no puedes hacer push.
\end{warningbox}

\subsubsection{Recomendación: ¿Dónde trabajar?}

Tienes tres opciones:

\begin{enumerate}
  \item \textbf{Opción 1: Trabajar SOLO en LaTeX-GitHub/} (recomendado)
  \begin{itemize}
    \item Editas archivos en \texttt{LaTeX-GitHub/}
    \item Haces \texttt{git add}, \texttt{git commit}, \texttt{git push} cuando quieras
    \item \texttt{03\_LaTex/} queda como respaldo antiguo (opcional)
  \end{itemize}

  \item \textbf{Opción 2: Borrar 03\_LaTex/ y trabajar solo en LaTeX-GitHub/}
  \begin{itemize}
    \item Libera espacio en disco
    \item Solo trabajas en un lugar
    \item Todo está respaldado en GitHub
  \end{itemize}

  \item \textbf{Opción 3: Mantener ambas carpetas}
  \begin{itemize}
    \item Trabajas en \texttt{LaTeX-GitHub/}
    \item \texttt{03\_LaTex/} permanece como respaldo local
    \item Debes copiar manualmente si quieres sincronizar (no recomendado)
  \end{itemize}
\end{enumerate}

\begin{infobox}[Resumen importante]
\begin{itemize}
  \item Git NO sincroniza automáticamente
  \item GitHub NO se actualiza automáticamente
  \item Debes hacer \texttt{git push} manualmente para subir cambios
  \item Carpetas diferentes son independientes entre sí
  \item \textbf{Recomendación:} Trabaja solo en \texttt{LaTeX-GitHub/}
\end{itemize}
\end{infobox}

%━━━━━━━━━━━━━━━━━━━━━━━━━━━━━━━━━━━━━━━━━━━━━━━━━━━━━━━━━━━━━━━━━━━━━━━━━━━━━
\section{Comandos Esenciales}
%━━━━━━━━━━━━━━━━━━━━━━━━━━━━━━━━━━━━━━━━━━━━━━━━━━━━━━━━━━━━━━━━━━━━━━━━━━━━━

\subsection{Ver el estado actual}

\begin{lstlisting}[style=bashstyle]
# Ver qué archivos cambiaron
git status
\end{lstlisting}

\textbf{Posibles salidas:}
\begin{itemize}
  \item \texttt{modified: archivo.tex} --- Archivo modificado
  \item \texttt{Untracked files} --- Archivos nuevos que Git no rastrea
  \item \texttt{Changes to be committed} --- Archivos listos para guardar
  \item \texttt{nothing to commit} --- Todo está guardado
\end{itemize}

\subsection{Agregar archivos al área de preparación}

\begin{lstlisting}[style=bashstyle]
# Agregar UN archivo específico
git add archivo.tex

# Agregar TODOS los archivos modificados
git add .

# Agregar múltiples archivos específicos
git add archivo1.tex archivo2.tex
\end{lstlisting}

\subsection{Crear un commit (guardar cambios)}

\begin{lstlisting}[style=bashstyle]
# Commit con mensaje descriptivo
git commit -m "Descripción breve del cambio"
\end{lstlisting}

\begin{warningbox}[Mensajes de commit efectivos]
\textbf{Buenos ejemplos:}
\begin{itemize}
  \item \texttt{"Agregada sección sobre funciones trigonométricas"}
  \item \texttt{"Corregido error en la ecuación de la línea 45"}
  \item \texttt{"Actualizada bibliografía con 3 nuevas referencias"}
\end{itemize}

\textbf{Malos ejemplos:}
\begin{itemize}
  \item \texttt{"cambios"} (muy vago)
  \item \texttt{"asdf"} (sin sentido)
  \item \texttt{"arreglado"} (¿qué arreglaste?)
\end{itemize}
\end{warningbox}

\subsection{Subir cambios a GitHub}

\begin{lstlisting}[style=bashstyle]
# Subir todos los commits al repositorio remoto
git push
\end{lstlisting}

\begin{infobox}[Nota importante]
\texttt{git push} requiere conexión a internet. Puedes hacer múltiples commits sin internet y luego hacer un solo \texttt{push} cuando te conectes.
\end{infobox}

\subsection{Ver el historial de cambios}

\begin{lstlisting}[style=bashstyle]
# Ver historial completo
git log

# Ver historial compacto (una línea por commit)
git log --oneline

# Ver últimos 5 commits
git log --oneline -5
\end{lstlisting}

\textbf{Ejemplo de salida:}
\begin{lstlisting}[style=bashstyle]
e72c314 Agregada nota final al taller
aa37424 Versión inicial del proyecto
\end{lstlisting}

\subsection{Ver diferencias entre versiones}

\begin{lstlisting}[style=bashstyle]
# Ver qué cambió en archivos modificados (no staged)
git diff

# Ver cambios en archivos en staging area
git diff --staged

# Ver cambios en un archivo específico
git diff archivo.tex
\end{lstlisting}

%━━━━━━━━━━━━━━━━━━━━━━━━━━━━━━━━━━━━━━━━━━━━━━━━━━━━━━━━━━━━━━━━━━━━━━━━━━━━━
\section{Trabajar con Repositorios de GitHub}
%━━━━━━━━━━━━━━━━━━━━━━━━━━━━━━━━━━━━━━━━━━━━━━━━━━━━━━━━━━━━━━━━━━━━━━━━━━━━━

\subsection{Clonar un repositorio desde GitHub}

\begin{lstlisting}[style=bashstyle]
# Clonar usando GitHub CLI
gh repo clone usuario/nombre-repositorio

# Ejemplo:
gh repo clone toribio99/LaTeX-Varios
\end{lstlisting}

Esto crea una carpeta con todo el proyecto y su historial completo.

\subsection{Crear un nuevo repositorio en GitHub}

\begin{lstlisting}[style=bashstyle]
# Crear repositorio público
gh repo create nombre-repositorio --public \
  --description "Descripción del proyecto"

# Ejemplo:
gh repo create MiTesisLatex --public \
  --description "Tesis de maestría en LaTeX"
\end{lstlisting}

\subsection{Ver repositorios remotos}

\begin{lstlisting}[style=bashstyle]
# Ver a dónde está conectado tu repositorio
git remote -v
\end{lstlisting}

\textbf{Salida típica:}
\begin{lstlisting}[style=bashstyle]
origin  https://github.com/toribio99/LaTeX-Varios.git (fetch)
origin  https://github.com/toribio99/LaTeX-Varios.git (push)
\end{lstlisting}

%━━━━━━━━━━━━━━━━━━━━━━━━━━━━━━━━━━━━━━━━━━━━━━━━━━━━━━━━━━━━━━━━━━━━━━━━━━━━━
\section{Flujo de Trabajo Completo: Ejemplo Práctico}
%━━━━━━━━━━━━━━━━━━━━━━━━━━━━━━━━━━━━━━━━━━━━━━━━━━━━━━━━━━━━━━━━━━━━━━━━━━━━━

\subsection{Escenario: Editaste un archivo}

Supongamos que editaste el archivo:\\
\texttt{TallerTrianguloRectangulo.tex}

\textbf{Ubicación del proyecto:}\\
\texttt{\textasciitilde/Documents/LaTeX-GitHub/LaTeX-Varios/}

\subsection{Paso 1: Navega al directorio}

\begin{lstlisting}[style=bashstyle]
cd ~/Documents/LaTeX-GitHub/LaTeX-Varios
\end{lstlisting}

\subsection{Paso 2: Verifica qué cambió}

\begin{lstlisting}[style=bashstyle]
git status
\end{lstlisting}

\textbf{Salida:}
\begin{lstlisting}[style=bashstyle]
Changes not staged for commit:
  modified:   Clases De Sheyra/Geometría analíca/TallerTrianguloRectangulo.tex
\end{lstlisting}

\subsection{Paso 3: Agrega el archivo modificado}

\begin{lstlisting}[style=bashstyle]
# Opción 1: Agregar ese archivo específico
git add "Clases De Sheyra/Geometría analíca/TallerTrianguloRectangulo.tex"

# Opción 2: Agregar todos los cambios
git add .
\end{lstlisting}

\subsection{Paso 4: Verifica el estado nuevamente}

\begin{lstlisting}[style=bashstyle]
git status
\end{lstlisting}

\textbf{Salida:}
\begin{lstlisting}[style=bashstyle]
Changes to be committed:
  modified:   Clases De Sheyra/Geometría analíca/TallerTrianguloRectangulo.tex
\end{lstlisting}

\subsection{Paso 5: Crea el commit}

\begin{lstlisting}[style=bashstyle]
git commit -m "Agregada nota final al taller de geometría analítica"
\end{lstlisting}

\textbf{Salida:}
\begin{lstlisting}[style=bashstyle]
[main e72c314] Agregada nota final al taller de geometría analítica
 1 file changed, 8 insertions(+), 4 deletions(-)
\end{lstlisting}

\subsection{Paso 6: Sube a GitHub}

\begin{lstlisting}[style=bashstyle]
git push
\end{lstlisting}

\textbf{Salida:}
\begin{lstlisting}[style=bashstyle]
To https://github.com/toribio99/LaTeX-Varios.git
   aa37424..e72c314  main -> main
\end{lstlisting}

\begin{successbox}[¡Listo!]
Tu cambio ahora está guardado en:
\begin{itemize}
  \item Tu computadora (Mac)
  \item GitHub (en la nube)
\end{itemize}

Puedes verlo en: \url{https://github.com/toribio99/LaTeX-Varios}
\end{successbox}

%━━━━━━━━━━━━━━━━━━━━━━━━━━━━━━━━━━━━━━━━━━━━━━━━━━━━━━━━━━━━━━━━━━━━━━━━━━━━━
\section{Trabajar Sin Internet}
%━━━━━━━━━━━━━━━━━━━━━━━━━━━━━━━━━━━━━━━━━━━━━━━━━━━━━━━━━━━━━━━━━━━━━━━━━━━━

\subsection{Git funciona 100\% local}

\textbf{Lo que SÍ puedes hacer sin internet:}
\begin{itemize}
  \item Editar tus archivos .tex
  \item Compilar con LaTeX
  \item \texttt{git add} (agregar archivos)
  \item \texttt{git commit} (guardar cambios)
  \item \texttt{git status} (ver estado)
  \item \texttt{git log} (ver historial)
  \item \texttt{git diff} (ver diferencias)
\end{itemize}

\textbf{Lo que NO puedes hacer sin internet:}
\begin{itemize}
  \item \texttt{git push} (subir a GitHub)
  \item \texttt{git pull} (descargar de GitHub)
  \item \texttt{gh repo create} (crear repositorio en GitHub)
  \item Clonar repositorios
\end{itemize}

\subsection{Flujo de trabajo sin internet}

\begin{infobox}[Escenario: Trabajas offline]
\textbf{Lunes (sin internet):}
\begin{lstlisting}[style=bashstyle]
# Editas archivo1.tex
git add archivo1.tex
git commit -m "Agregué introducción"

# Editas archivo2.tex
git add archivo2.tex
git commit -m "Completé capítulo 2"
\end{lstlisting}

\textbf{Martes (con internet):}
\begin{lstlisting}[style=bashstyle]
# Subes TODOS los commits que hiciste offline
git push
\end{lstlisting}

Resultado: Los 2 commits se suben a GitHub de una vez.
\end{infobox}

%━━━━━━━━━━━━━━━━━━━━━━━━━━━━━━━━━━━━━━━━━━━━━━━━━━━━━━━━━━━━━━━━━━━━━━━━━━━━━
\section{Recuperar Versiones Anteriores}
%━━━━━━━━━━━━━━━━━━━━━━━━━━━━━━━━━━━━━━━━━━━━━━━━━━━━━━━━━━━━━━━━━━━━━━━━━━━━━

\subsection{Ver historial}

\begin{lstlisting}[style=bashstyle]
# Ver lista de commits
git log --oneline
\end{lstlisting}

\textbf{Ejemplo de salida:}
\begin{lstlisting}[style=bashstyle]
e72c314 Agregada nota final al taller
aa37424 Versión inicial del proyecto
9d82076 Primer commit
\end{lstlisting}

\subsection{Recuperar un archivo específico}

\begin{lstlisting}[style=bashstyle]
# Recuperar archivo de un commit específico
git checkout aa37424 archivo.tex
\end{lstlisting}

\begin{warningbox}[¡Atención!]
Este comando \textbf{sobrescribe} el archivo actual con la versión antigua. Asegúrate de hacer commit de tus cambios actuales primero, o los perderás.
\end{warningbox}

\subsection{Ver un archivo sin modificarlo}

\begin{lstlisting}[style=bashstyle]
# Ver contenido de archivo en un commit específico
git show aa37424:ruta/al/archivo.tex
\end{lstlisting}

Esto solo \textbf{muestra} el contenido, no modifica tu archivo.

%━━━━━━━━━━━━━━━━━━━━━━━━━━━━━━━━━━━━━━━━━━━━━━━━━━━━━━━━━━━━━━━━━━━━━━━━━━━━━
\section{Archivo .gitignore para LaTeX}
%━━━━━━━━━━━━━━━━━━━━━━━━━━━━━━━━━━━━━━━━━━━━━━━━━━━━━━━━━━━━━━━━━━━━━━━━━━━━━

\subsection{¿Qué es .gitignore?}

Un archivo especial que le dice a Git qué archivos \textbf{NO} rastrear.

\subsection{¿Por qué necesitamos .gitignore en LaTeX?}

LaTeX genera muchos archivos auxiliares (\texttt{.aux}, \texttt{.log}, \texttt{.synctex.gz}, etc.) que:
\begin{itemize}
  \item Cambian cada vez que compilas
  \item No son necesarios para el control de versiones
  \item Hacen que Git muestre cambios innecesarios
\end{itemize}

\subsection{Contenido típico de .gitignore para LaTeX}

\begin{lstlisting}[style=bashstyle]
# Archivos auxiliares de LaTeX
*.aux
*.log
*.out
*.toc
*.lof
*.lot
*.fls
*.fdb_latexmk
*.synctex.gz
*.bbl
*.blg
*.idx
*.ilg
*.ind
*.listing

# PDFs (opcional - descomenta si quieres versionar PDFs)
# *.pdf

# Archivos de sistema
.DS_Store
*~
*.swp
__MACOSX/

# Otros
general.idx
*.zip
\end{lstlisting}

\subsection{Crear .gitignore}

\begin{lstlisting}[style=bashstyle]
# Crear archivo .gitignore en la raíz del proyecto
cd ~/Documents/LaTeX-GitHub/MiProyecto
nano .gitignore
# (pega el contenido de arriba y guarda)
\end{lstlisting}

%━━━━━━━━━━━━━━━━━━━━━━━━━━━━━━━━━━━━━━━━━━━━━━━━━━━━━━━━━━━━━━━━━━━━━━━━━━━━━
\section{Solución de Problemas Comunes}
%━━━━━━━━━━━━━━━━━━━━━━━━━━━━━━━━━━━━━━━━━━━━━━━━━━━━━━━━━━━━━━━━━━━━━━━━━━━━━

\subsection{Olvidé hacer commit antes de editar}

\textbf{Problema:} Editaste un archivo pero quieres volver a la versión anterior.

\textbf{Solución:}
\begin{lstlisting}[style=bashstyle]
# Descartar cambios no guardados
git restore archivo.tex
\end{lstlisting}

\subsection{Hice commit pero no quería hacerlo}

\textbf{Problema:} Hiciste commit de cambios que no querías guardar.

\textbf{Solución (antes de push):}
\begin{lstlisting}[style=bashstyle]
# Deshacer el último commit (mantiene cambios)
git reset --soft HEAD~1

# Deshacer el último commit (descarta cambios)
git reset --hard HEAD~1
\end{lstlisting}

\begin{warningbox}[¡Cuidado con --hard!]
\texttt{git reset --hard} \textbf{elimina permanentemente} tus cambios. Úsalo solo si estás seguro.
\end{warningbox}

\subsection{Error al hacer push}

\textbf{Problema:} \texttt{git push} falla con error.

\textbf{Posibles causas y soluciones:}
\begin{enumerate}
  \item \textbf{Sin conexión a internet}
    \begin{lstlisting}[style=bashstyle]
# Espera a tener internet y vuelve a intentar
git push
    \end{lstlisting}

  \item \textbf{Cambios en GitHub que no tienes localmente}
    \begin{lstlisting}[style=bashstyle]
# Descargar cambios primero
git pull
# Luego subir
git push
    \end{lstlisting}
\end{enumerate}

\subsection{No sé en qué estado está mi repositorio}

\textbf{Solución:}
\begin{lstlisting}[style=bashstyle]
# Ver estado completo
git status

# Ver últimos commits
git log --oneline -5

# Ver diferencias no guardadas
git diff
\end{lstlisting}

%━━━━━━━━━━━━━━━━━━━━━━━━━━━━━━━━━━━━━━━━━━━━━━━━━━━━━━━━━━━━━━━━━━━━━━━━━━━━━
\section{Comandos de Referencia Rápida}
%━━━━━━━━━━━━━━━━━━━━━━━━━━━━━━━━━━━━━━━━━━━━━━━━━━━━━━━━━━━━━━━━━━━━━━━━━━━━━

\begin{table}[h!]
\centering
\small
\begin{tabular}{|l|p{8cm}|}
\hline
\textbf{Comando} & \textbf{Descripción} \\
\hline
\texttt{git status} & Ver estado actual del repositorio \\
\hline
\texttt{git add .} & Agregar todos los cambios \\
\hline
\texttt{git add archivo.tex} & Agregar archivo específico \\
\hline
\texttt{git commit -m "msg"} & Guardar cambios con mensaje \\
\hline
\texttt{git push} & Subir commits a GitHub \\
\hline
\texttt{git pull} & Descargar cambios de GitHub \\
\hline
\texttt{git log --oneline} & Ver historial compacto \\
\hline
\texttt{git diff} & Ver cambios no guardados \\
\hline
\texttt{git restore archivo} & Descartar cambios no guardados \\
\hline
\texttt{gh auth status} & Ver estado de autenticación \\
\hline
\texttt{gh repo create} & Crear repositorio en GitHub \\
\hline
\texttt{gh repo clone} & Clonar repositorio \\
\hline
\end{tabular}
\caption{Comandos Git y GitHub más usados}
\end{table}

%━━━━━━━━━━━━━━━━━━━━━━━━━━━━━━━━━━━━━━━━━━━━━━━━━━━━━━━━━━━━━━━━━━━━━━━━━━━━━
\section{Flujo de Trabajo Recomendado}
%━━━━━━━━━━━━━━━━━━━━━━━━━━━━━━━━━━━━━━━━━━━━━━━━━━━━━━━━━━━━━━━━━━━━━━━━━━━━━

\subsection{Rutina diaria}

\begin{successbox}[Recomendación: Commits frecuentes]
\textbf{Al empezar a trabajar:}
\begin{lstlisting}[style=bashstyle]
cd ~/Documents/LaTeX-GitHub/MiProyecto
git status  # Ver si hay cambios pendientes
git pull    # Actualizar (si trabajas desde varias computadoras)
\end{lstlisting}

\textbf{Durante el trabajo:}
\begin{itemize}
  \item Haz commits cada vez que completes una sección importante
  \item No esperes a que todo esté ``perfecto''
  \item Un commit por cada cambio lógico
\end{itemize}

\textbf{Al terminar la sesión:}
\begin{lstlisting}[style=bashstyle]
git add .
git commit -m "Descripción de lo que hiciste hoy"
git push  # Si tienes internet
\end{lstlisting}
\end{successbox}

\subsection{Frecuencia de commits}

\textbf{Buenas prácticas:}
\begin{itemize}
  \item \textbf{Muy frecuente}: Cada cambio significativo (ej: completar una sección)
  \item \textbf{Moderado}: Al final de cada sesión de trabajo
  \item \textbf{Mínimo}: Al menos una vez al día si trabajaste
\end{itemize}

\textbf{No recomendado:}
\begin{itemize}
  \item Una vez por semana (demasiado tiempo entre commits)
  \item Solo cuando terminas el proyecto completo
  \item Cada pequeño cambio de una palabra
\end{itemize}

%━━━━━━━━━━━━━━━━━━━━━━━━━━━━━━━━━━━━━━━━━━━━━━━━━━━━━━━━━━━━━━━━━━━━━━━━━━━━━
\section{Organización de Repositorios LaTeX}
%━━━━━━━━━━━━━━━━━━━━━━━━━━━━━━━━━━━━━━━━━━━━━━━━━━━━━━━━━━━━━━━━━━━━━━━━━━━━━

\subsection{Estructura recomendada}

Para proyectos LaTeX grandes, organiza por temas:

\begin{lstlisting}[style=bashstyle]
LaTeX-GitHub/
├── LaTeX-Varios/              # Trabajos misceláneos
├── LaTeX-Libros-Terminados/   # Libros completos
├── LaTeX-Practicas/           # Ejercicios y prácticas
├── LaTeX-Fisica/              # Proyectos de física
└── LaTeX-Matematicas/         # Proyectos de matemáticas
\end{lstlisting}

\subsection{Cada repositorio debe contener}

\begin{itemize}
  \item \texttt{.gitignore} --- Lista de archivos a ignorar
  \item Archivos \texttt{.tex} --- Tus documentos fuente
  \item Carpetas de figuras/imágenes
  \item \texttt{README.md} (opcional) --- Descripción del proyecto
\end{itemize}

%━━━━━━━━━━━━━━━━━━━━━━━━━━━━━━━━━━━━━━━━━━━━━━━━━━━━━━━━━━━━━━━━━━━━━━━━━━━━━
\section{Conclusión}
%━━━━━━━━━━━━━━━━━━━━━━━━━━━━━━━━━━━━━━━━━━━━━━━━━━━━━━━━━━━━━━━━━━━━━━━━━━━━━

\subsection{Beneficios de usar Git + GitHub}

\begin{successbox}[¡Nunca más perderás trabajo!]
Con Git y GitHub:
\begin{itemize}
  \item Tu trabajo está seguro en la nube
  \item Puedes recuperar cualquier versión anterior
  \item Trabajas con confianza sabiendo que todo está respaldado
  \item Colaboras fácilmente con otras personas
  \item Accedes a tus proyectos desde cualquier computadora
\end{itemize}
\end{successbox}

\subsection{Los 3 comandos que usarás el 90\% del tiempo}

\begin{lstlisting}[style=bashstyle]
git add .
git commit -m "Descripción del cambio"
git push
\end{lstlisting}

\subsection{Recursos adicionales}

\begin{itemize}
  \item Documentación oficial de Git: \url{https://git-scm.com/doc}
  \item GitHub CLI: \url{https://cli.github.com}
  \item Tu perfil en GitHub: \url{https://github.com/toribio99}
\end{itemize}

\vspace{1cm}
\begin{center}
\Large\textbf{¡Feliz versionado!}

\medskip
\normalsize
Documento creado el \today

\smallskip
\textit{Toribio de J. Arrieta F.}
\end{center}

\end{document}
